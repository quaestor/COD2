\section*{Aufgabe 3 - Permutationen als unitäre Transformationen}
\addcontentsline{toc}{subsection}{Aufgabe 3 - Permutationen als unitäre Transformationen}
Für eine Matrix $T_{\pi}$ vom Format $n \times n$ (siehe Aufgabenstellung) mit
Eigenvektor $w$ zum Eigenwert $\lambda$ gilt:
\[ T w = \lambda w \]
Ist $\pi$ nun eine zyklische Permutation, so erfüllen die $n$ Vektoren
\[ w_k = [e^{m\cdot 2 \pi i \frac{k}{n}}]_{1\leq m\leq n} \quad (1 \leq k \leq n) \]
diese Gleichung, da eine Multiplikation dieses Vektors mit $\lambda := e^{l\cdot 2 \pi
i \frac{k}{n}}$ diese um $l$ Stellen zyklisch permutieren. Damit ist $\lambda$ auch der
entsprechende Eigenwert, der erwartungsgemäß den Absolutbetrag $1$ besitzt
(unitäre Transformation).

Die Vektoren $w_k$ sind linear unabhängig (vgl. \textsc{Vandermonde}-Matrix!),
also bilden die Vektoren
\[ u_k = \frac{w_k}{\left\|w_k\right\|} \]
eine ON-Basis von Eigenvektoren.

Die Erweiterung auf beliebige nicht-zyklische Permutation ist ohne Weiteres
machbar. Da jede Permutation durch disjunkte zyklische Permutationen
darstellbar ist, kann die Matrix $T_{\pi}$ durch ein Produkt von Matrizen
dargestellt werden, dessen Faktoren jeweils zyklische Permutationen auf
disjunkten Teilmengen $Q$ von $\{1, 2, \dots, n\}$ sind. Die Eigenvektoren der
einzelnen Matrizen sind die $w_k = [e^{m\cdot 2 \pi i \frac{k}{n}}]_{1\leq
m\leq \#Q}\ (1\leq k \leq \#Q)$ mit eingeschobenen Nullen in den an der
Permutation unbeteiligten Komponenten und für diese zusätzlich noch die
kanonischen Einheitsvektoren der jeweiligen Komponente.

Konkrete Durchführung an der Permutation
\begin{center}
\begin{tabular}{r|cccccccc}
$v_i$ & $v_1$ & $v_2$ & $v_3$ & $v_4$ & $v_5$ & $v_6$ & $v_7$ & $v_8$ \\
\hline
$\mathcal{T}v_i$ & $v_1$ & $v_3$ & $v_5$ & $v_7$ & $v_8$ & $v_4$ & $v_6$ & $v_2$
\end{tabular}
\end{center}

Die einzelnen zyklischen Permutationen sind
\begin{center}
\begin{tabular}{r|c}
$v_i$ & $v_1$ \\
\hline
$\mathcal{T}_1v_i$ & $v_1$
\end{tabular}\hspace{1em}
\begin{tabular}{r|cccc}
$v_i$ & $v_2$ & $v_3$ & $v_5$ & $v_8$ \\
\hline
$\mathcal{T}_2v_i$ & $v_3$ & $v_5$ & $v_8$ & $v_2$
\end{tabular}\hspace{1em}
\begin{tabular}{r|ccc}
$v_i$ & $v_4$ & $v_6$ & $v_7$ \\
\hline
$\mathcal{T}_3v_i$ & $v_7$ & $v_4$ & $v_6$
\end{tabular}
\end{center}

Die entsprechenden Eigenvektoren sind somit ($\omega_3 = e^{\frac{2}{3} \pi i}$)
\begin{center}
\begin{tabular}{rclccccccccr}
$w_1$ & $=$ & $[$ & $1$ & $0$ & $0$ & $0$ & $0$ & $0$ & $0$ & $0$ & $]$ \\
\hline
$w_2$ & $=$ & $[$ & $0$ & $1$  & $1$ & $0$ & $0$ & $0$ & $0$ & $0$ & $]$ \\
$w_3$ & $=$ & $[$ & $0$ & $-1$ & $1$ & $0$ & $1$ & $0$ & $0$ & $-1$ & $]$ \\
$w_4$ & $=$ & $[$ & $0$ & $i$  & $-1$ & $0$ & $-i$ & $0$ & $0$ & $1$ & $]$ \\
$w_5$ & $=$ & $[$ & $0$ & $-i$ & $-1$ & $0$ & $i$ & $0$ & $0$ & $1$ & $]$ \\
\hline
$w_6$ & $=$ & $[$ & $0$ & $0$ & $0$ & $1$ & $0$ & $1$ & $1$ & $0$ & $]$ \\
$w_7$ & $=$ & $[$ & $0$ & $0$ & $0$ & $\omega_3$ & $0$ & $\omega_3^2$ & $1$ & $0$ & $]$ \\
$w_8$ & $=$ & $[$ & $0$ & $0$ & $0$ & $\overline{\omega_3}$ & $0$ & $\overline{\omega_3}^2$ & $1$ & $0$ & $]$
\end{tabular}
\end{center}
