\documentclass[a4paper,svgnames]{article}
\usepackage[utf8]{inputenc}
\usepackage{ngerman}
\usepackage{a4wide}
\usepackage{fancyhdr}
\usepackage{tikz}
\usepackage{enumerate}
\usepackage{listings}
\usepackage{color}
\usepackage[colorlinks=true,linkcolor=black,a4paper=true]{hyperref}
\usetikzlibrary{calc}
\usetikzlibrary{matrix}
\usetikzlibrary{positioning}
\usepackage{dsfont}
\usepackage{amsmath, amsthm, amssymb}

\title{Information und Codierung II\\
       Sommersemester 2010 \\
       \small Lösungen zu den Übungsaufgaben}
\author{\LaTeX-Version von Christoph Rauch\\ FAU Erlangen-Nürnberg}

\lhead{Christoph Rauch}
\rhead{\today}
%\newcommand*\tikzhead{%
%  \begin{tikzpicture}[remember picture,overlay]
%    \node[yshift=-2cm] at (current page.north west)
%      {\begin{tikzpicture}[remember picture, overlay, decoration=Koch snowflake]
%        \draw (2.5cm,-0.95cm) -- (\paperwidth/2-1.95cm,-0.95cm);
%        \draw decorate{ decorate{ decorate{ decorate{ (\paperwidth/2-1.95cm,-0.95cm) -- (\paperwidth/2+2cm,-0.95cm) }}}};
%        \draw (\paperwidth/2+2cm,-0.95cm) -- (\paperwidth-2.5cm,-0.95cm);
%       \end{tikzpicture}
%      };
%   \end{tikzpicture}}
%\fancyhf{}
%\renewcommand*\headrulewidth{0pt}
%\lhead{Christoph Rauch}
%\chead{\tikzhead}
%\rhead{\today}
%\pagestyle{fancy}
%\newcommand*\headcolor{Aqua}
%\newcommand*\setheadcolor[1]{\renewcommand*\headcolor{#1}}

\makeatletter
% Different format for headings
\def\section{\@startsection{section}{1}%
  \z@{.7\baselineskip\@plus\baselineskip}{.5\baselineskip}%
  {\large\scshape\centering}}
% This does spacing around caption.
\setlength{\abovecaptionskip}{0.5em}
\setlength{\belowcaptionskip}{0.5em}
% This does justification (left) of caption.
\long\def\@makecaption#1#2{%
  \vskip\abovecaptionskip
  \sbox\@tempboxa{#2}%
  \ifdim \wd\@tempboxa >\hsize
    #2\par
  \else
    \global \@minipagefalse
    \hb@xt@\hsize{\box\@tempboxa\hfill}%
  \fi
  \vskip\belowcaptionskip}
\makeatother

\newcommand{\hk}{\mathcal{H}_{k}}
\newcommand{\hkp}{\mathcal{H}_{k+1}}

\newcommand{\bra}{\langle}
\newcommand{\ket}{\rangle}
\newcommand{\bk}[1]{\bra #1 \ket}
\newcommand{\vtwo}[2]{\begin{bmatrix}#1\\#2\end{bmatrix}}
\newcommand{\twobeg}{\begin{minipage}{0.45\textwidth}}
\newcommand{\twomid}{\end{minipage}\hfill\begin{minipage}{0.45\textwidth}}
\newcommand{\twoend}{\end{minipage}}

\begin{document}
%\maketitle
%\tableofcontents
\pagestyle{fancy}
%\section*{Aufgabe 1 - Pauli-Matrizen}
\addcontentsline{toc}{subsection}{Aufgabe 1 - Pauli-Matrizen}
\[ X = \vtwo{0&1}{1&0} \quad Y = \vtwo{0&-i}{i&0} \quad Z = \vtwo{1&0}{0&-1} \]
\begin{enumerate}
        \item $\ $

                \twobeg
                \begin{eqnarray*}
                   X\cdot Y &=& \vtwo{i&0}{0&-i} = i\cdot Z = -Y\cdot X \\
                   X\cdot Z &=& \vtwo{0&-1}{1&0} = -i\cdot Y = -Z\cdot X \\
                   Y\cdot Z &=& \vtwo{0&i}{i&0} = i\cdot X = -Z\cdot Y \\
                \end{eqnarray*}\twomid
                \begin{eqnarray*}
                   \left[X, Y\right] &=& i\cdot Z + i\cdot Z = 2i\cdot Z \\
                   \left\{X, Y\right\} &=& i\cdot Z - i\cdot Z = 0 \\
                   \left[X, Z\right] &=& -i\cdot Y - i\cdot Y = -2i\cdot Y \\
                   \left\{X, Z\right\} &=& -i\cdot Y + i\cdot Y = 0 \\
                   \left[Y, Z\right] &=& i\cdot X + i\cdot X = 2i\cdot X \\
                   \left\{Y, Z\right\} &=& i\cdot X - i\cdot X = 0
                \end{eqnarray*}\twoend
        \item
                        $X$ und $Z$ sind symmetrische, reelle Matrizen, also offensichtlich selbstadjungiert
                        %
                        \begin{center}
                        \begin{tabular}{lr}
                        $Y^{\dag} = \vtwo{0&i}{-i&0}^T = Y\quad \quad$&
                        $\quad \quad XX^{\dag} = YY^{\dag} = ZZ^{\dag} = \vtwo{1&0}{0&1} = \mathds{1}$\end{tabular}
                        \end{center}
                        %
                        Folglich sind die \textsc{Pauli}-Matrizen selbstadjungierte, unitäre Matrizen.
        \item $\ $

                \twobeg
                Eigenwerte:
                \begin{eqnarray*}
                        \det(X-\lambda\mathds{1}) &=& \det\begin{pmatrix}-\lambda&1\\1&-\lambda\end{pmatrix} = \lambda^2 - 1 \\
                        \det(Y-\lambda\mathds{1}) &=& \det\begin{pmatrix}-\lambda&-i\\i&-\lambda\end{pmatrix} = \lambda^2 - 1 \\
                        \det(Z-\lambda\mathds{1}) &=& \det\begin{pmatrix}1-\lambda&0\\0&1-\lambda\end{pmatrix} = \lambda^2 - 1
                \end{eqnarray*}
                \[ \Rightarrow \lambda = \pm 1 \]
                \twomid
              Eigenvektoren (berechnet über $Mv = \pm v, \ M \in \{X, Y, Z\}$):

                        \begin{center}
                        \begin{tabular}{r|c|c|c}
                                $\lambda$ & $X$ & $Y$ & $Z$ \\
                                \hline
                                &&&\\
                                $1$& $\vtwo{1}{1}$ & $\vtwo{1}{i}$ & $\vtwo{1}{0}$ \\[0.6em]
                                &&&\\
                                $-1$& $\vtwo{1}{-1}$ & $\vtwo{1}{-i}$ & $\vtwo{0}{1}$
                        \end{tabular}
                        \end{center}
                \twoend
        \item Für eine Matrix $A = a_{ij} \in \mathcal{M}_2(\mathds{C}) \ (1
              \leq i, j \leq 2)$ gilt in der gegebenen Darstellung, dass die Komponenten
              $a_{11}$ und $a_{22}$ durch $a_0 \mathds{1} + a_z Z$ sowie $a_{12}$ und
              $a_{21}$ durch $a_xX + a_yY$ gegeben sind. Die Diagonalen bzw. Nebendiagonalen
              von jeweils $\mathds{1}$ und $Z$ sowie $X$ und $Y$ sind offensichtlich linear
              unabhängig und bilden daher eine Basis des $\mathds{C}^2$. Daher ist eine
              Darstellung jeder beliebigen Matrix aus $\mathcal{M}_2(\mathds{C})$ auf diese
              weise möglich und auch eindeutig.
        \item Die Eigenwerte der \textsc{Hadamard}-Matrix sind $\lambda = \pm
              1$ (unitär!) mit den Eigenvektoren $(1~+~\sqrt{2}, 1)^T, (1-\sqrt{2}, 1)^T$.
              \[ H = \frac{1}{\sqrt{2}} (X + Z) \]
              \begin{eqnarray*}
              H\cdot X\cdot H &=& \frac{1}{\sqrt{2}} (X\cdot X + Z\cdot X)
              \cdot H = \frac{1}{2} (X\cdot X\cdot X + X\cdot X\cdot Z + Z\cdot X\cdot X +
              Z\cdot X\cdot Z) = \\
              &&\frac{1}{2} (X + Z + Z + Z\cdot X \cdot Z) = \frac{1}{2} (X
              + 2Z + (-i)\cdot Z\cdot Y) = Z \\
              H\cdot Z\cdot H &=& H \cdot H \cdot X \cdot H \cdot H = X \\
              H\cdot Y\cdot H &=& i H\cdot X\cdot Z\cdot H = i H\cdot X \cdot H \cdot H \cdot Z \cdot H = i Z\cdot X = -Y\cdot X\cdot X = -Y
              \end{eqnarray*}
        \item \[ H\cdot Z\cdot H = H\cdot Z^{\frac{1}{2}}\cdot H\cdot H\cdot Z^{\frac{1}{2}}\cdot H = X^{\frac{1}{2}}\cdot X^{\frac{1}{2}} \]
              \[ H\cdot Z^{\frac{1}{2}}\cdot H = H\cdot \vtwo{1&0}{0&i}\cdot H = \frac{1}{2}\vtwo{1+i&1-i}{1-i&1+i} = X^{\frac{1}{2}} \]
\end{enumerate}

%\section*{Aufgabe 2 - Unschärfeprinzip für hermitesche Matrizen}
\addcontentsline{toc}{subsection}{Aufgabe 2 - Unschärfeprinzip für hermitesche Matrizen}
\begin{enumerate}
        \item Für hermitesche Matrizen $A$ und $B$ gilt: 
        \[ \bra BA \ket = \bra B^{\dag}A^{\dag} \ket = \bra (AB)^{\dag} \ket = \bra AB \ket^* \]
        Zudem ist \[ | a |^2 = a \cdot a^* \]
        Durch einfaches Ausmultiplizieren ergibt sich:
        \begin{eqnarray*}
        |\bra [A,B] \ket|^2 + |\bra \{A,B\} \ket|^2 &=& |\bra AB \ket - \bra AB \ket^*|^2 
             + |\bra AB \ket + \bra AB \ket^*|^2 = \\
        &=& (\bra AB \ket - \bra AB \ket^*)\cdot(\bra AB \ket - \bra AB \ket^*)^* \\
        &+& (\bra AB \ket + \bra AB \ket^*)\cdot(\bra AB \ket + \bra AB \ket^*)^* = \\
        &=& (\bra AB \ket \bra AB \ket^* - \bra AB \ket^2 - (\bra AB \ket^*)^2
        + \bra AB \ket^* \bra AB \ket) \\ &+& (\bra AB \ket \bra AB \ket^* + \bra AB \ket^2
        + (\bra AB \ket^*)^2 + \bra AB \ket^* \bra AB \ket) \\[0.5em]
        &=& 4 \bra AB \ket \bra AB \ket^* = 4 |\bra A\cdot B \ket|^2 
        \end{eqnarray*}
        $\hfill\square$
        \item Hier wird die Beziehung $\bra v | A | v \ket = \bra A^{\dag}v | v
        \ket = \bra Av | v \ket$ für hermitesche Matrizen $A$ ausgenutzt:
        \begin{eqnarray*}
        |\bra AB \ket|^2 &=& |\bra v | AB | v \ket |^2 = |\bra A^{\dag}v | Bv \ket|^2 = \\
        |\bra Av | Bv \ket|^2 &\overset{CSU}{\leq}&
        \left\|Av\right\|^2\cdot\left\|Bv\right\|^2 = \bra Av | Av \ket \cdot \bra Bv |
        Bv \ket = \bra A^2 \ket \bra B^2 \ket
        \end{eqnarray*}
        Damit gilt
        \begin{equation}
                |\bra[A,B]\ket|^2 \leq 4\bra A^2 \ket \bra B^2 \ket
        \end{equation}
        \item
        Vorbemerkung: 
        \begin{eqnarray*}
                [A - \bra A \ket, B - \bra B \ket] &=& (A - \bra B \ket)(B -
                \bra B \ket) - (B - \bra B \ket)(A - \bra A \ket) = \\
                &=& AB - A\bra B \ket - \bra A \ket B + \bra A \ket \bra B \ket
                \\&-& BA + B \bra A \ket + \bra B \ket A - \bra B \ket \bra A \ket = \\
                &=& AB - BA = [A, B]
        \end{eqnarray*}
        \begin{eqnarray*}
        (\Delta A)^2\cdot(\Delta B)^2= \bra (A - \bra A \ket)^2\ket \cdot \bra
        (B - \bra B \ket)^2\ket \overset{(1)}{\geq} \frac{1}{4} |\bra[A - \bra A \ket,
        B - \bra B \ket]\ket|^2 = \frac{1}{4} |\bra[A, B]\ket|^2
        \end{eqnarray*}
        Daraus folgt
        \[ \Delta A \cdot \Delta B \geq \frac{1}{2} |\bra[A,B]\ket| \]$\hfill\square$
\end{enumerate}

\section*{Aufgabe 3 - Permutationen als unitäre Transformationen}
\addcontentsline{toc}{subsection}{Aufgabe 3 - Permutationen als unitäre Transformationen}
Für eine Matrix $T_{\pi}$ vom Format $n \times n$ (siehe Aufgabenstellung) mit
Eigenvektor $w$ zum Eigenwert $\lambda$ gilt:
\[ T w = \lambda w \]
Ist $\pi$ nun eine zyklische Permutation, so erfüllen die $n$ Vektoren
\[ w_k = [e^{m\cdot 2 \pi i \frac{k}{n}}]_{1\leq m\leq n} \quad (1 \leq k \leq n) \]
diese Gleichung, da eine Multiplikation dieses Vektors mit $\lambda := e^{l\cdot 2 \pi
i \frac{k}{n}}$ diese um $l$ Stellen zyklisch permutieren. Damit ist $\lambda$ auch der
entsprechende Eigenwert, der erwartungsgemäß den Absolutbetrag $1$ besitzt
(unitäre Transformation).

Die Vektoren $w_k$ sind linear unabhängig (vgl. \textsc{Vandermonde}-Matrix!),
also bilden die Vektoren
\[ u_k = \frac{w_k}{\left\|w_k\right\|} \]
eine ON-Basis von Eigenvektoren.

Die Erweiterung auf beliebige nicht-zyklische Permutation ist ohne Weiteres
machbar. Da jede Permutation durch disjunkte zyklische Permutationen
darstellbar ist, kann die Matrix $T_{\pi}$ durch ein Produkt von Matrizen
dargestellt werden, dessen Faktoren jeweils zyklische Permutationen auf
disjunkten Teilmengen $Q$ von $\{1, 2, \dots, n\}$ sind. Die Eigenvektoren der
einzelnen Matrizen sind die $w_k = [e^{m\cdot 2 \pi i \frac{k}{n}}]_{1\leq
m\leq \#Q}\ (1\leq k \leq \#Q)$ mit eingeschobenen Nullen in den an der
Permutation unbeteiligten Komponenten und für diese zusätzlich noch die
kanonischen Einheitsvektoren der jeweiligen Komponente.

Konkrete Durchführung an der Permutation
\begin{center}
\begin{tabular}{r|cccccccc}
$v_i$ & $v_1$ & $v_2$ & $v_3$ & $v_4$ & $v_5$ & $v_6$ & $v_7$ & $v_8$ \\
\hline
$\mathcal{T}v_i$ & $v_1$ & $v_3$ & $v_5$ & $v_7$ & $v_8$ & $v_4$ & $v_6$ & $v_2$
\end{tabular}
\end{center}

Die einzelnen zyklischen Permutationen sind
\begin{center}
\begin{tabular}{r|c}
$v_i$ & $v_1$ \\
\hline
$\mathcal{T}_1v_i$ & $v_1$
\end{tabular}\hspace{1em}
\begin{tabular}{r|cccc}
$v_i$ & $v_2$ & $v_3$ & $v_5$ & $v_8$ \\
\hline
$\mathcal{T}_2v_i$ & $v_3$ & $v_5$ & $v_8$ & $v_2$
\end{tabular}\hspace{1em}
\begin{tabular}{r|ccc}
$v_i$ & $v_4$ & $v_6$ & $v_7$ \\
\hline
$\mathcal{T}_3v_i$ & $v_7$ & $v_4$ & $v_6$
\end{tabular}
\end{center}

Die entsprechenden Eigenvektoren sind somit ($\omega_3 = e^{\frac{2}{3} \pi i}$)
\begin{center}
\begin{tabular}{rclccccccccr}
$w_1$ & $=$ & $[$ & $1$ & $0$ & $0$ & $0$ & $0$ & $0$ & $0$ & $0$ & $]$ \\
\hline
$w_2$ & $=$ & $[$ & $0$ & $1$  & $1$ & $0$ & $0$ & $0$ & $0$ & $0$ & $]$ \\
$w_3$ & $=$ & $[$ & $0$ & $-1$ & $1$ & $0$ & $1$ & $0$ & $0$ & $-1$ & $]$ \\
$w_4$ & $=$ & $[$ & $0$ & $i$  & $-1$ & $0$ & $-i$ & $0$ & $0$ & $1$ & $]$ \\
$w_5$ & $=$ & $[$ & $0$ & $-i$ & $-1$ & $0$ & $i$ & $0$ & $0$ & $1$ & $]$ \\
\hline
$w_6$ & $=$ & $[$ & $0$ & $0$ & $0$ & $1$ & $0$ & $1$ & $1$ & $0$ & $]$ \\
$w_7$ & $=$ & $[$ & $0$ & $0$ & $0$ & $\omega_3$ & $0$ & $\omega_3^2$ & $1$ & $0$ & $]$ \\
$w_8$ & $=$ & $[$ & $0$ & $0$ & $0$ & $\overline{\omega_3}$ & $0$ & $\overline{\omega_3}^2$ & $1$ & $0$ & $]$
\end{tabular}
\end{center}

\section*{Aufgabe 4 - Sicheres Unterscheiden von Zuständen}
\addcontentsline{toc}{subsection}{Aufgabe 4 - Sicheres Unterscheiden von Zuständen}
\begin{enumerate}
        \item In der Spektraldarstellung lassen sich die $E_i$ darstellen als
                \[ E_i = \sum_{\lambda} \lambda\ |v_{\lambda} \ket \bra v_\lambda| \]
              Wird eine Funktion $f$ auf $E_i$ angewendet, so entspricht dies
              in der Spektraldarstellung der Anwendung von $f$ auf die Eigenwerte $\lambda$:
                \[ \sqrt{E_i} = \sum_{\lambda} \sqrt{\lambda}\ |v_{\lambda}\ket \bra v_\lambda| \]
              Da die $E_i$ positiv sind ist dies wohldefiniert und die
              $\sqrt{\lambda}$ sind reell und positiv. Wegen $\bra \psi_i | E_j | \psi_i \ket
              = \delta_{i,j}$ gilt (positiv beinhaltet selbstadjungiert!)
                \[ \bra \psi_1 | E_2 | \psi_1 \ket = \bra \psi_1 | \sqrt{E_2}
                \sqrt{E_2} | \psi_1 \ket = \bra \sqrt{E_2}\psi_1 | \sqrt{E_2}\psi_1 \ket =
                \left\|\sqrt{E_2}\psi_1\right\|^2 = 0 \]
              und damit
                \[ \sqrt{E_2}\psi_1 = 0 \]
              sowie
                \[ \bra \phi | E_2 | \psi_1 \ket = \bra \phi | \sqrt{E_2} | \sqrt{E_2}\psi_1 \ket = 0 \]
        \item Aufgrund von $E_1 + E_2 = \mathds{1}$ und da $| \phi \ket$ normiert ist gilt:
                \[ 1 = \bra \phi | \mathds{1}  | \phi \ket =
                \bra \phi | E_1+E_2 | \phi \ket = \underbrace{\bra \phi | E_1 | \phi \ket}_{\geq 0} +
                \underbrace{\bra \phi | E_2 | \phi \ket}_{\geq 0} \]
                \[ \Rightarrow \bra \phi | E_i | \phi \ket \leq 1 \]
        \item Durch Ausmultiplizieren des Skalarprodukts $\bra \psi_2 | E_2 | \psi_2 \ket$ erhält man
                \[ \bra\psi_2|E_2|\psi_2\ket = |\alpha|^2
                \underbrace{\bra\psi_2|E_2|\psi_1\ket}_{=0} + \beta^*\alpha
                \underbrace{\bra\phi|E_2|\psi_1\ket}_{=0} + \alpha^*\beta
                \underbrace{\bra\psi_1|E_2|\phi\ket}_{=\bra \phi| E_2 | \psi_1\ket^* = 0^* = 0} + |\beta|^2
                \underbrace{\bra\phi|E_2|\phi\ket}_{\leq 1} \]
                \begin{eqnarray*}
                 &\Updownarrow& \\
                1 &=& |\beta|^2 \bra\phi|E_2|\phi\ket \\[1em]
                \Rightarrow |\beta|^2 &=& 1,\\\alpha &=& 0
                \end{eqnarray*}
        Daher ist $|\psi_2\ket = |\phi\ket$ und, da $|\phi\ket \perp
        |\psi_1\ket$ nach Annahme gilt, natürlich auch $|\psi_2\ket \perp \psi_1\ket$.
\end{enumerate}

\end{document}
