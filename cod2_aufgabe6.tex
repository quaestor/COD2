\section*{Aufgabe 6 - Dreidimensionale reelle Darstellung von 1-qubit-Zuständen}
\addcontentsline{toc}{subsection}{Aufgabe 6 - Dreidimensionale reelle Darstellung von 1-qubit-Zuständen}
\begin{enumerate}
\item Mit der Darstellung \[ \vket{\psi} \leftrightarrow \vec\psi =
        \begin{bmatrix}x\\y\\z\end{bmatrix} =
        \begin{bmatrix}\cos\phi \sin\theta \\
                      \sin\phi \sin\theta \\
                      \cos \theta
        \end{bmatrix} \]
      und den \textsc{Pauli}-Matrizen 
        \[ X = \vtwo{0&1}{1&0},\ Y = \vtwo{0&-i}{i&0},\ Z = \vtwo{1&0}{0&-1} \]
      erhält man für $\vket{\psi}\vbra{\psi}$:
        \[ \vket{\psi}\vbra{\psi} =
        \vtwo{\cos\frac{\theta}{2}}{e^{i\phi}\sin\frac{\theta}{2}}\cdot
        \begin{bmatrix}{\cos\frac{\theta}{2}} &
        {e^{-i\phi}\sin\frac{\theta}{2}}\end{bmatrix} =
        \vtwo{ \cos^2\frac{\theta }{2} & e^{-i \phi } \cos\frac{\theta }{2}
        \sin\frac{\theta }{2}}{ e^{i \phi } \cos\frac{\theta }{2} \sin\frac{\theta }{2}
        & \sin^2\frac{\theta }{2}} = \]
        \[ = \vtwo{\frac{1}{2}(1+\cos\theta) & \frac{1}{2} e^{-i \phi }
        \sin\theta}{ \frac{1}{2}e^{i\phi}\sin\theta & \frac{1}{2}(1-\cos\theta) } = 
        \vtwo{\frac{1}{2}(1+\cos\theta) & \frac{1}{2}(\cos\phi
        \sin\theta - i\sin\phi\sin\theta)}{ \frac{1}{2}(\cos\phi\sin\theta +
        i\sin\phi\sin\theta) & \frac{1}{2}(1-\cos\theta) } = \]
        \[ = \frac{1}{2}\cdot \mathds{1} + \frac{1}{2}\cdot \cos\phi\sin\theta\cdot X + \frac{1}{2}\cdot \sin\phi\sin\theta\cdot Y + \frac{1}{2}\cdot \cos\theta\cdot Z = \]
        \[ = \frac{1}{2}\left(\mathds{1} + x\cdot X + y\cdot Y + z\cdot Z\right) \]
\newcommand{\twopsi}{\vtwo{\cos\frac{\theta}{2}}{e^{i\phi}\sin\frac{\theta}{2}}}
\item Der Erwartungswert für eine Messung mittels einer selbstadjungierten Matrix $M$ ist
        \[ \vbra{\psi}M\vket{\psi} \]
      und damit für die Pauli-Matrizen $\sigma_x = X, \sigma_y = Y, \sigma_z = Z$:
        \begin{alignat*}{2}
        \vbra{\psi}X\vket{\psi} &= e^{-i \phi} \cos\frac{\theta }{2}
        \sin\frac{\theta }{2}+e^{i \phi} \cos\frac{\theta}{2} \sin\frac{\theta}{2} = \cos\phi\sin\theta &= x \\
        \vbra{\psi}Y\vket{\psi} &=
        i e^{-i \phi } \cos \frac{\theta }{2} \sin \frac{\theta }{2} - i e^{i \phi }
        \cos \frac{\theta }{2} \sin\frac{\theta }{2} = \sin\phi\sin\theta &=  y \\
        \vbra{\psi}Z\vket{\psi} &= \cos^2\frac{\theta}{2}-\sin^2\frac{\theta}{2} = \cos\theta &= z
        \end{alignat*}
\item man kann leicht nachrechnen, dass die Eigenvektoren von $R_x, R_y, R_z$
      identisch mit den Eigenvektoren von $X, Y, Z$ sind. Die Eigenwerte (durch
      Nachrechnen bestimmt) und damit die Spektraldarstellung ergibt sich zu:
        \[ R_j(\delta) = e^{-i\frac{d}{2}} \vket{v_{j,0}}\vbra{v_{j,0}} + e^{i\frac{d}{2}} \vket{v_{j,1}}\vbra{v_{j,1}} \]
      mit $v_{j,0}, v_{j,1}$ den entsprechenden Eigenvektoren von $\sigma_j$.

      Daraus ist ersichtlich, dass die $A_j,\ \ j \in \{x, y, z\}$ für die
      Darstellung in der Form $R_j(\delta) = e^{-i\frac{\delta}{2} A_j}$ genau die
      \textsc{Pauli}-Matrizen sein müssen:
        \[ A_x = X, \quad A_y = Y, A_z = Z \]
\item Die Rotationsmatrix für eine Drehung um die Achse $v$ ($|v| = 1$) ist gegeben durch:
        \[ R_v(\delta) = v_x R_x(\delta) \cdot v_y R_y(\delta) \cdot v_z R_z(\delta) \]
      beziehungsweise nach Aufgabenteil 3.:
        \[ R_v(\delta) = e^{-i\frac{\delta}{2}Q}, \quad Q = v_x X + v_y Y + v_z Z \]
\end{enumerate}
