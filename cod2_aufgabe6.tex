\section*{Aufgabe 6 - Dreidimensionale reelle Darstellung von 1-qubit-Zuständen}
\addcontentsline{toc}{subsection}{Aufgabe 6 - Dreidimensionale reelle Darstellung von 1-qubit-Zuständen}
\begin{enumerate}
\item Mit der Darstellung \[ \vket{\psi} \leftrightarrow \vec\psi =
        \begin{bmatrix}x\\y\\z\end{bmatrix} =
        \begin{bmatrix}\cos\phi \sin\theta \\
                      \sin\phi \sin\theta \\
                      \cos \theta
        \end{bmatrix} \]
      und dem Vektor-Operator der \textsc{Pauli}-Matrizen 
        \[ X = \vtwo{0&1}{1&0},\ Y = \vtwo{0&-i}{i&0},\ Z = \vtwo{1&0}{0&-1},\quad \vec\sigma = \begin{bmatrix}X\\Y\\Z\end{bmatrix} \]
      erhält man für $\vket{\psi}\vbra{\psi}$:
        \[ \vket{\psi}\vbra{\psi} =
        \vtwo{\cos\frac{\theta}{2}}{e^{i\phi}\sin\frac{\theta}{2}}\cdot
        \begin{bmatrix}{\cos\frac{\theta}{2}} &
        {e^{-i\phi}\sin\frac{\theta}{2}}\end{bmatrix} =
        \vtwo{ \cos^2\frac{\theta }{2} & e^{-i \phi } \cos\frac{\theta }{2}
        \sin\frac{\theta }{2}}{ e^{i \phi } \cos\frac{\theta }{2} \sin\frac{\theta }{2}
        & \sin^2\frac{\theta }{2}} = \]
        \[ = \vtwo{\frac{1}{2}(1+\cos\theta) & \frac{1}{2} e^{-i \phi }
        \sin\theta}{ \frac{1}{2}e^{i\phi}\sin\theta & \frac{1}{2}(1-\cos\theta) } = 
        \vtwo{\frac{1}{2}(1+\cos\theta) & \frac{1}{2}(\cos\phi
        \sin\theta - i\sin\phi\sin\theta)}{ \frac{1}{2}(\cos\phi\sin\theta +
        i\sin\phi\sin\theta) & \frac{1}{2}(1-\cos\theta) } = \]
        \[ = \frac{1}{2}\cdot \mathds{1} + \frac{1}{2}\cdot \cos\phi\sin\theta\cdot X + \frac{1}{2}\cdot \sin\phi\sin\theta\cdot Y + \frac{1}{2}\cdot \cos\theta\cdot Z =
        \frac{1}{2}(\mathds{1} +
        \vec\psi^t \cdot\vec\sigma) \]
\end{enumerate}
