\section*{Aufgabe 5 - qubit-Zustände und Transformationen}
\addcontentsline{toc}{subsection}{Aufgabe 5 - qubit-Zustände und Transformationen}
\begin{enumerate}
\item Sei $\hat\beta_{ij} = \sqrt{2} \cdot \beta_{ij}$, dann gilt:
        \[ \hat\beta_{00} = \vket{0} \otimes \vket{0} + \vket{1} \otimes \vket{1} = \vtwo{1}{0} \otimes \vtwo{1}{0} + \vtwo{0}{1} \otimes \vtwo{0}{1} = \vtwo{1\\0}{0\\1} \]
        und analog
        \[ \hat\beta_{01} = \vtwo{0\\1}{1\\0}, \quad \hat\beta_{10} = \vtwo{1\\0}{0\\-1}, \quad \hat\beta_{11} = \vtwo{0\\-1}{1\\0} \]
        Die $\hat\beta_{ij}$ besitzen jeweils zwei Komponenten mit Betrag $1$, daher sind die $\beta_{ij}$ normalisiert. Offensichtlich gilt auch
        \[ \beta_{ij} \cdot \beta_{kl} = \delta_{ik} \delta_{jl} \]
        Damit bilden die \textsc{Bell}-Zustände eine Orthonormalbasis des Raumes $\mathds{C}^2 \otimes \mathds{C}^2$.
\item Dies kann man durch Induktion beweisen:
        \begin{itemize}
        \item Induktionsanfang: $n = 1$
                \[ H^{\otimes 1} = H = \vtwo{1&1}{1&-1} = \frac{1}{\sqrt{2^0}} \left[(-1)^{bin(x)\cdot bin(y)}\right]_{0\leq x,y < 2} \]
        \item Induktionsvoraussetzung: die Annahme gelte für ein $n \in \mathds{N}$
        \item Induktionsschritt:
                \[ H^{\otimes n+1} = H^{\otimes n} \otimes H \overset{IV}{=}
                   \frac{1}{\sqrt{2^n}} \left[(-1)^{bin(x)\cdot bin(y)}\right]_{0\leq x,y < 2^n}
                   \otimes H = \frac{1}{2^{n+1}} \vtwo{A&B}{C&-D} \]
                mit Blockmatrizen
                \begin{align*}
                A &= \left[(-1)^{bin(x)\cdot bin(y)}\right]_{0\leq x,y < 2^n} &
                B &= \left[(-1)^{bin(x)\cdot bin(y)}\right]_{\substack{2^n \leq  x < 2^{n+1} \\
                                                                              0  \leq y < 2^n}} \\
                C &= \left[(-1)^{bin(x)\cdot bin(y)}\right]_{\substack{0   \leq x < 2^n \\
                                                                              2^n \leq y < 2^{n+1}}} &
                D &= \left[(-1)^{bin(x)\cdot bin(y)}\right]_{2^n \leq x,y < 2^{n+1}}
                \end{align*}
                Es gilt (mit $bin(x) = $ Binärdarstellung von $x$ mit $n+1$ Bits)
                \[ bin(a)\cdot bin(b) = bin(2^n + a)\cdot bin(b) = bin(a)\cdot bin(2^n + b) \]
                sowie
                \[ bin(2^n + a)\cdot bin(2^n + b) = bin(a)\cdot bin(b) + 1 \]
                also ist $A = B = C = -D$ und damit
                \[ H^{\otimes n+1} = \frac{1}{\sqrt{2^{n+1}}} \left[(-1)^{bin(x)\cdot bin(y)}\right]_{0 \leq x < 2^{n+1}} \]$\hfill\square$
        \end{itemize}
        Der Basisvektor $\vket{0}_n$ enstpricht in Vektordarstellung dem
        kanonischen Einheitsvektor der Länge $2^n$ der ersten Komponente. Von rechts an
        $H^{\otimes n}$ multipliziert erhält man also die erste Spalte von $H^{\otimes n}$.
        Dies ist der Vektor $\mathbf{1} = \frac{1}{\sqrt{2^n}} (1, 1, \dots, 1)$ der Länge $2^n$. In \textsc{Dirac}-Notation:
        \[ H^{\otimes n} \vket{0}_n = \frac{1}{\sqrt{2^n}} \sum_{i=0}^{2^n-1} \vket{i}_n = \frac{1}{\sqrt{2^n}} \left(\vket{0} + \vket{1}\right)^{\otimes n} \]
        Analog erhält man für $\vket{1}$ die letzte Spalte von $H^{\otimes n}$.
        Dies ist der Vektor, der durch das $n$-fache Tensorprodukt der zweiten Spalte
        von $H$ entsteht, also in \textsc{Dirac}-Notation:
        \[ H^{\otimes n} \vket{1}_n = \frac{1}{\sqrt{2^n}} (\vket{0} - \vket{1})^{\otimes n} \]
\item Die Äquivalenz kann durch schlichtes Nachrechnen gezeigt werden:
        \[ (H \otimes H) \circ C_{01} \circ (H \otimes H) = \vtwo{1&0&0&0\\0&0&0&1}{0&0&1&0\\0&1&0&0} \]
       Dies ist genau $C_{10}$, denn der rechte Schaltkreis permutiert die
       Basisvektoren folgendermaßen:
       \begin{eqnarray*}
               \vket{00} \mapsto \vket{00} \\
               \vket{01} \mapsto \vket{11} \\
               \vket{10} \mapsto \vket{10} \\
               \vket{11} \mapsto \vket{01}
       \end{eqnarray*}
\end{enumerate}
